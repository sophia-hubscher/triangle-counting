\documentclass[12pt]{report}
\usepackage[utf8]{inputenc}
\usepackage{graphicx}  % For including figures
\usepackage{amsmath}   % For math symbols
\usepackage{amssymb}   % For math symbols
\usepackage{hyperref}  % For clickable links
\usepackage{geometry}  % For setting margins
\usepackage{setspace}  % For setting line spacing
\usepackage{fancyhdr}  % For customizing headers
\usepackage{cite}      % For bibliography
\usepackage{lipsum}    % For placeholder text
\usepackage{titlesec}  % For section formatting

\geometry{letterpaper, margin=1in}
\setstretch{1.5}  % 1.5 line spacing

\begin{document}

\section{Introduction}

Counting triangles is a fundamental problem in graph theory with widespread applications in social networks, bioinformatics, and more.
These triangles, formed by three mutually connected nodes, can, in social network graphs, represent closed friendships, indicating a high level of local connectivity, which can give great insite into the network as a whole.
However, for large graphs, especially sparse ones, where the number of edges is much smaller compared to the number of possible edges, efficiently counting these triangles poses significant computational challenges.

To address this, we turn toward randomized algorithms. These algorithms rely on probabilistic techniques to sample subsets of the graph's nodes, estimating the global triangle count based on local observations.
By doing this, these algorithms significantly reduce the runtime compared to exact methods, particularly in sparse graphs where only a fraction of potential edges exist.
Additionally, randomized algorithms often provide tunable accuracy, allowing for a trade-off between precision and performance, making them ideal for processing large-scale networks.




\cite{knuth1984texbook}


% Bibliography
\newpage
\bibliographystyle{plain}
\bibliography{thesis_bib}

\end{document}
